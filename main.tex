\DocumentMetadata{
  lang        = en-US,
  pdfstandard = ua-2,
  pdfstandard = a-4f, %or a-4
  tagging=on,
  tagging-setup={math/setup={mathml-SE,mathml-AF}},
  testphase={phase-III,table,math,firstaid}
}
\documentclass{article}
% \usepackage[accsupp]{axessibility}
\input{acc_header}
% \input{tex/color_thm_head}


\usepackage[
    pdftitle={Accessibility Testing},
    pdfauthor={Chuck Rocca},
    linkcolor=blue!70!black,
    colorlinks=true,
    urlcolor=blue!70!black
]{hyperref}
\usepackage{unicode-math}
% \usepackage[tagged]{accessibility}

%%%% Use to Adjust Fontsize %%%%
% \fontsize{14pt}{16pt}
% \selectfont

\title{Accessibility Notes}
\author{Charles Rocca}
\date{November 2024}

\begin{document}

\maketitle

\section{Exponents and Exponential Functions}
\begin{expository}[Exponent Rules]\label{expo:exponent rules}
    \begin{itemize}
        \item When \(n\in\mathbb{Z}\), \(a^n\) is the product of \(n\)-factors of \(a\).
        \item \(a^n\cdot a^m=a^{n+m}\)
        \item \(a^n/a^m = a^{n-m}\) when \(a\neq 0\)
        \item \(a^{-n}=1/a^n\)
        \item \(a^{1/n}=\sqrt[n]{a}\)
        \item \((ab)^n=a^n\cdot b^n\)
        \item \((a/b)^n=a^n/b^n\) when \(b\neq 0\)
        \item \(a^0=1\) when \(a\neq 0\)
        \item \(a^1=a\)
    \end{itemize}
\end{expository}

\begin{defn}[Exponential Function]
    Given a positive constant \(a\) and a non-zero constant \(k\) the function \(f(x)=ka^x\) is called an \emph{exponential function}.  Exponential functions are characterized by the fact that there values have a constant ratio, in particular \(f(x+1)/f(x)=a\) for all \(x\).
\end{defn}

\begin{figure}[h]
    \centering
    \includegraphics[width=0.5\linewidth,alt={graph of an exponential growth function going to 0 in the negative direction and infinity in the positive}]{images/Basic_plot.png}
    \caption{Plot of Exponential Growth}
    \label{fig:exp growth}
\end{figure}

\section{Systematic Testing of Equations}


\section{Generated Images}

\section{Images from GitHub/Jupyter Notebook}

\begin{figure}[h]
    \centering
    \includegraphics[
        width=0.75\linewidth,
        alt={hyperbola openning to the left and right as well as the asymptotes to the hyperbola}
    ]{images/Sample_hyperbola.png}
    \caption{Hyperbola \(\frac{x^2}{a^2}-\frac{y^2}{b^2}=1\) and Asymptotes \(y=\pm\frac{b}{a}\)}
    \label{fig:placeholder}
\end{figure}

\end{document}



